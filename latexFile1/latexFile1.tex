\documentclass{article}

%adding package like this:
%	\usepackage{PACKAGENAME}

% equation% is located in this package
\usepackage{amsmath}

\usepackage{graphicx}
\usepackage{subcaption}
\usepackage{setspace}

\title{My first document}
\date{2013-09-01}
\author{Saleh Afzoon}

\setcounter{tocdepth}{3}

\begin{document}

%delete first page number
\pagenumbering{gobble}
\maketitle

\newpage
  
\pagenumbering{arabic}

\tableofcontents
\newpage


\section{Text Section}
Hello World!

\subsection{Subsection}
Structuring a document is easy!

\subsubsection{Subsubsection}
More text.

\paragraph{Paragraph}
some more text.

\subparagraph{Subparagraph}
Even more text.

\section{Math Section}
inline math sample : $ f(x) = x^3 +1 $ \\

and this is math equation environment sample:
\begin{equation*}
	f(x) = x^2
\end{equation*}


and align environment sample:
\begin{align*}
  f(x) &= x^2\\
  g(x) =& \frac{1}{x}\\
  h(x) = \frac{x^2}{\sqrt[3]{x}} \\
  F(x) = \int^a_b \frac{1}{3}x^3 &
\end{align*}

sample matix : 
$$
\left[
\begin{matrix}
1 & 0 & 2 & 3\\
0 & 1 & 4 & 1
\end{matrix}
\right]
\left(\frac{1}{\sqrt{x}}\right)
$$

%label which is invisible, but useful if we want to refer to our figure in our document.You can use the \ref command to refer to the figure (marked by label)
\newpage

\section{Image Section}
\begin{figure}[h!]

\centering
\includegraphics[width=0.7\linewidth]{profile2.jpg}
\caption{cute cat}
\label{fig:cat}
\end{figure}

Figure \ref{fig:cat} shows a boat.

\section{Multi Image Section}
\begin{figure}[h!]
\centering
\begin{subfigure}[b]{0.4\linewidth}
\includegraphics[width=\linewidth]{profile2.jpg}
\caption{cat1.}
\end{subfigure}
\begin{subfigure}[b]{0.4\linewidth}
\includegraphics[width=\linewidth]{profile2.jpg}
\caption{cat2.}
\end{subfigure}

\end{figure}

\newpage

\begin{appendix}
  \listoffigures
  \listoftables
\end{appendix}

\end{document}

%	article , book 
%	article -> add page number auto
% \begin \end -> defines an environment. An environment is simply an area of your document where certain typesetting rules apply. 

% Useful settings for pagenumbering:
%	gobble - no numbers
%	arabic - arabic numbers
%	roman - roman numbers


% align environment for multiple equations and automatic alignment

%options for figure
%h (here) - same location
%t (top) - top of page
%b (bottom) - bottom of page
%p (page) - on an extra page
%! (override) - will force the specified location


%\setcounter{tocdepth}{1} % Show sections
%\setcounter{tocdepth}{2} % + subsections
%\setcounter{tocdepth}{3} % + subsubsections
%\setcounter{tocdepth}{4} % + paragraphs
%\setcounter{tocdepth}{5} % + subparagraphs